\documentclass[a4paper,oneside,article]{memoir}
\usepackage[utf8]{inputenc}
\usepackage[T1]{fontenc}
\usepackage[danish]{babel}
\renewcommand\danishhyphenmins{22}
\usepackage[osf]{mathpazo}
\linespread{1.06}
\usepackage{microtype}
\usepackage{tket}
\usepackage{lipsum}
\nofiles

% Fancy break med tre små stjerner, der signalerer et skifte, der ikke
% er start nok til at skulle have egen opgavebetegnelse.
\usepackage{fourier-orns}
\newcommand{\starbreak}{%
\fancybreak{\starredbullet\quad\starredbullet\quad\starredbullet}}

% Hvis kun der er en side i dokumentet, er der ingen grund til at vise
% sidetallet. Derfor kan vi med lidt magi skjule det i netop det
% tilfælde.
\AtEndDocument{\ifnum\value{page} = 1\thispagestyle{empty}\fi}

\newcommand{\testtext}{%
  \begin{center}
    \begin{tabular}{ *{8}{c }}
      \toprule
      FORM & \KASS & INKA & SEKR & PR & \CERM & \VC & nf \\
      \ikonFORM & \ikonKASS & \ikonINKA & \ikonSEKR & \ikonPR & \ikonCERM & \ikonVC & \ikonNF \\
      \bottomrule
    \end{tabular} \\
    Køller: \ikonKASSkoeller \qquad\qquad Kugleramme: \ikonKASSkugler
  \end{center}}

\begin{document}
\author{Steffen Videbæk Fredsgaard}
\title{En lille test}
\date{\today}
\maketitle


Integer vel velit efficitur, ornare felis a, vulputate \ikonFORM
justo. Proin sollicitudin \ikonINKA ante sed felis \ikonCERM euismod,
in sollicitudin lectus \ikonKASS efficitur. Etiam eget diam nisl. Cras
a nisl venenatis, gravida turpis eu, bibendum \ikonNF mi. Pellentesque
feugiat, elit \ikonPR vitae ornare posuere, mi dolor maximus sem,
lacinia ultricies felis purus eget \ikonVC arcu. Sed mollis
consectetur urna sit amet commodo. Pellentesque ex ante, cursus eu
ullamcorper eu, efficitur \ikonSEKR id est. Sed luctus elit ex, et
malesuada massa mollis sed. Cras lacinia consectetur neque, quis
fringilla nunc tempus et. Maecenas sed sodales ex. Fusce vitae leo
pulvinar, consequat urna sed, sodales nisl. Phasellus consectetur in
augue ac rhoncus. Duis eleifend cursus ante, non blandit eros
efficitur nec.


\chapter{Standard i dag}

\testtext


\chapter{TKsetup til 2013}

\TKsetup{gf=2013}
\testtext


\chapter{2013, men med kolleKASS=nix}

\TKsetup{koelleKASS=nix}
\testtext


\end{document}

%%% Local Variables: 
%%% mode: latex
%%% TeX-master: t
%%% End: 
