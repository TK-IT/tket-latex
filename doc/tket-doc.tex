% [Emacs] Time-stamp: <2011-06-03 16:04:51 villemoes> 
% File: tket-doc.tex
% Documentation of tket.sty

\documentclass[a4paper,article,oneside,danish]{memoir}

\usepackage[utf8]{inputenc}
\usepackage[T1]{fontenc}
\usepackage[danish]{babel}
\usepackage[danish=guillemets]{csquotes}

% textcomp til \textdollaroldstyle og \pounds. bbm til \mathbbm. De bliver brugt
% i eksempler til konfiguration.
\usepackage{textcomp}
\usepackage{bbm}

\usepackage{graphicx}
\usepackage[table]{xcolor}

\usepackage{lmodern}
\usepackage{microtype}
\usepackage{multirow}

\usepackage{tket}
% Parse versionen af tket som lige er blevet hentet.
\makeatletter
\def\parse@tket@version#1/#2/#3 #4: #5\@nil{%
  \def\pTKETYear{#1}%
  \def\pTKETMonth{#2}%
  \def\pTKETDay{#3}%
  \def\pTKETVer{#4}%
  \def\pTKETDesc{#5}
}
\expandafter\expandafter\expandafter\parse@tket@version\csname ver@tket.sty\endcsname\@nil
\makeatother


% Use \vref, \Vref or some other macro from varioref for references.

% varioref, hyperref, cleverref and bookmarks need to be loaded in this order.
% See cleverref doc. section 13.
\usepackage{varioref}
\usepackage{hyperref}
%%%% Hyperref-opsætning
\definecolor{nicered}{rgb}{.647,.129,.149}
\hypersetup{colorlinks=true,
  urlcolor=nicered,
  bookmarksopen=true,
  pdfauthor={Rasmus Villemoes},
  pdftitle={Pakken tket},
  unicode=true
}
\usepackage{cleveref}


%%%% Diverse rare småting
\newcommand{\emaillink}[1]{\href{mailto:#1}{#1}}

\newcommand{\pakkenavn}[1]{\textsf{#1}}
\newcommand{\ptket}{\pakkenavn{tket}\xspace}
\newcommand{\filnavn}[1]{\texttt{#1}}

\definecolor{option}{rgb}{.7,.05,.15}
\newcommand*{\descopt}[1]{%
  \medskip\noindent\llap{\color{option}#1\hspace*{8pt}}\ignorespaces}
\newcommand*{\optionname}[1]{\textcolor{option}{#1}}

% Til brug i tabellen som viser \TKxyz og venner...
\newcommand{\vismakro}[2][]{\cmdprint{#2} & #1#2}


%%%% Kode- og resultatfremvisning
\usepackage{mdframed}
\usepackage{minted}
\setminted[latex]{fontsize=\small}
\mdfdefinestyle{kodestyle}{backgroundcolor=orange!10, hidealllines=true}
\surroundwithmdframed[style=kodestyle]{minted}
\newcommand{\mdframedinputminted}[2]{\begin{mdframed}[style=kodestyle]\inputminted{#1}{#2}\end{mdframed}}

\newmdenv[backgroundcolor=blue!7, hidealllines=true]{resultat}

\title{Pakken \ptket (\pTKETVer)}
\author{Rasmus Villemoes}
\date{\today}

\hyphenation{krum-spring}

\begin{document}

\maketitle

\tableofcontents*

\chapter{Indledning}
\label{cha:indledning}

Dette er dokumentationen for \LaTeX-pakken \ptket, som kan bruges til
at lave forskellige \TK-relaterede ting i \LaTeX. Pakkens oprindelse
er ukendt; dens oprindelige formål var at give let adgang til at
skrive \hbox{TÅGEKAMMERET} med de såkaldte hoppe-danse-bogstaver,
\TKET. Siden er pakken blevet udvidet med mange andre
funktionaliteter. Blandt andet er der nu en
anciennitetspræfiksberegner, og (modificerede udgaver af) de
piktogrammer der nu om dage findes på bestyrelsens (og FU'ernes)
T-shirts er også inkluderet. Se også \vref{sec:historik}.

\chapter{Demonstration}
\label{cha:demonstration}

En del eksempler på hvad du kan med pakken ses af nedenstående
eksempel.
\mdframedinputminted{latex}{eksempel-demo.tex}

\begin{resultat}
\TKET ser sådan ud hvis det skrives på \RemToR. \RemToR er \TKETs
skrivemaskine \ikonNF, men \ikonNF er egentlig bare et piktogram der
skal symbolisere næstformanden. \KASS har også sådan et, \ikonKASS.
Dem der blev \ikonFU-bærere i 2004 kan i dag kalde sig
\TKprefix{2004}FU'er. \TKsetup{gf=2008}I 2008 var de
\TKprefix{04}FU'er. \TKsetup{dollar=\textsterling,gf=*}Nu er vi
tilbage ved \TKgetgf-valgperioden. \TKprefix{2000}\KASS foretrækker
at skrive sin titel sådan. Hvis man vil følge Aarhus Kommunes eksempel
kan man skrive \TKAA og \TKETsAA etc.

\end{resultat}

For at vide mere er du nødt til at læse videre.

\chapter{Installation}
\label{cha:installation}

Den letteste måde at installere pakken er at få fat i zip-filen
\filnavn{tket.tds.zip}, som indeholder alle de nødvendige filer. Den
skal du pakke ud i roden af et \enquote{lokalt texmf-træ}. Hvor og hvordan du
finder den mappe afhænger af din \TeX-distribution (\mbox{\TeX{}
  Live}/\mbox{Mac\TeX}/\mbox{MiK\TeX}).

\section{\TeX{} Live}
\label{sec:tex-live}

Hvis du bruger
\href{http://www.tug.org/texlive/doc/texlive-en/texlive-en.html#x1-380003.4.7}{\TeX{}
  Live} på Windows er den rigtige mappe formentlig noget i retning af
\meta{\%USERPROFILE\%}{\ttfamily \textbackslash texmf}; på et
linux-agtigt system er det formentlig \nolinkurl{~/texmf}. Du kan
forvisse dig om den rigtige værdi ved at køre kommandoen {\ttfamily
  kpsewhich -var-value TEXMFHOME} i en terminal.

Når du har pakket zip-filen ud skal fildatabasen opdateres. Det gøres
lettest ved at udføre kommandoen {\ttfamily mktexlsr}. På Windows er
der måske en grafisk måde at gøre det på. 

\section{Mac\TeX}
\label{sec:mactex}

Mac\TeX{} er dybest set bare \TeX{} Live pakket til Mac OSX. Den
rigtige mappe er
\href{http://www.tug.org/mactex/faq/#qm04}{formentlig} i dette
tilfælde \nolinkurl{~/Library/texmf}.

\section{MiK\TeX}
\label{sec:miktex}

Keine Ahnung.

\chapter{Brug af pakken}
\label{cha:brug-af-pakken}

I sin enkleste form benyttes pakken ved simpelthen at skrive

\begin{minted}{latex}
\usepackage{tket}
\end{minted}
et sted i ens preamble. Pakken forudsætter at pakkerne
\pakkenavn{xspace}, \pakkenavn{graphicx}, \pakkenavn{etoolbox} og
\pakkenavn{xkeyval} er til stede (men dem har du højst sandsynligt
allerede; ellers er det på høje tid at opdatere din
\TeX-installation). Hvis du vil give \enquote{options} til
\pakkenavn{graphicx}-pakken skal du, af \TeX niske årsager, hente
denne pakke \emph{før} du henter \ptket-pakken (efter en pakke er
hentet kan man nemlig ikke hente den igen med nye options).

\section{Options}
\label{sec:options}

Pakken accepterer et valgfrit argument, som skal være en liste af
såkaldte \emph{key-value}-par. Syntaksen er den samme som for makroen
\cs{TKsetup} beskrevet i \vref{sec:konfiguration}. Med
andre ord er

\begin{minted}[escapeinside=||]{latex}
\usepackage[|\meta{key-value-liste}|]{tket}
\end{minted}
helt ækvivalent til

\begin{minted}[escapeinside=||]{latex}
\usepackage{tket}
\TKsetup{|\meta{key-value-liste}|}
\end{minted}


\chapter{Brugerkommandoer}
\label{cha:brugerkommandoer}

\section{\texorpdfstring{\cs{TKET}}{\textbackslash TKET} og venner}
\label{sec:cstket-og-venner}


\Vref{tab:simple-makroer} viser de simple forkortelsesmakroer som
pakken stiller til rådighed. 

\begin{table}[hbtp]
  \centering
  \caption{Oversigt over simple makroer fra \ptket.}
  \label{tab:simple-makroer}
  \begin{tabularx}{.9\textwidth}{l X l X}
    \vismakro{\TK}       & \vismakro{\TKET}    \\
    \vismakro{\TKETs}    & \vismakro{\TKETS}   \\
    \vismakro{\TKAA}     & \vismakro{\TKETAA}  \\
    \vismakro{\TKETsAA}  & \vismakro{\TKETSAA} \\  
    \vismakro{\RemToR}   & \vismakro{\KASS}    \\
    \vismakro{\CERM}     & \vismakro{\VC}      \\
    \cs{TKurl} & \multicolumn{3}{l}{\TKurl}
  \end{tabularx}
\end{table}

Definitionerne af disse makroer indeholder alle den ganske snedige
\cs{xspace} (fra pakken \pakkenavn{xspace}) i enden, og kan derfor
bruges umiddelbart i almindelig tekst uden at man skal tilføje tomme
tuborgpar \texttt{\{\}} eller gøre andre krumspring. Hvis du ikke ved
hvad dette betyder, så se enten \pakkenavn{xspace}-dokumentationen
eller følgende eksempel.

\mdframedinputminted{latex}{eksempel-xspace.tex}
Med \pakkenavn{xspace}-funktionalitet giver dette
\begin{resultat}
\TKET, eller i daglig tale »kammeret«, sælger altid øl, og \KASS er
ond. \TKETs skrivemaskine hedder \RemToR. På \TKETs hjemmeside \TKurl
er der billeder.

\end{resultat}  
Uden dette ville vi have fået
{% This is a hack to deactivate the \xspace mechanism. Don't do this
 % at home...
\let\xspace\relax %
\begin{resultat}
\TKET, eller i daglig tale »kammeret«, sælger altid øl, og \KASS er
ond. \TKETs skrivemaskine hedder \RemToR. På \TKETs hjemmeside \TKurl
er der billeder.

\end{resultat}
}
\noindent Med andre ord opfører makroerne sig som de skal. (Bemærk at
der \emph{ikke} indsættes mellemrum når makroen efterfølges af fx et
komma eller punktum.)

\section{Ikoner}
\label{sec:ikoner}

Piktogrammerne, kendt fra BESTs og FU'ernes T-shirts, er tilgængelige
via makroerne vist i \vref{tab:ikon}
\newcolumntype{Y}{>{\hsize=1.2\hsize\hfill}X<{\hfill\strut}}
\newcolumntype{Z}{>{\hsize=0.6\hsize}X}
\begin{table}[hbt]
  \centering
  \caption{\TK-ikoner.}
  \label{tab:ikon}
  \begin{tabularx}{.7\textwidth}{l Y Z l Y}
    \vismakro{\ikonFORM} && \vismakro{\ikonNF} \\
    \vismakro{\ikonCERM} && \vismakro{\ikonVC} \\
    \vismakro{\ikonSEKR} && \vismakro{\ikonPR} \\
    \vismakro{\ikonKASS} && \vismakro{\ikonINKA} \\
    \vismakro{\ikonFU}
  \end{tabularx}
\end{table}

Faktisk er ovenstående ikke helt de billeder der findes på
T-shirtsene. De er nemlig ikke i sig selv egnede til at blive skaleret
ned til en størrelse så de kan indgå som symboler i løbende tekst --
stregtykkelsen er for lille, og mange detaljer bliver nærmest
usynlige. De forenklede versioner, som vi kalder ikoner, i
\vref{tab:ikon} blev oprindelig lavet til brug i regelhæftet til
\enquote{Kammerludo}.

\subsection{\KASSpdf of INKA ikoner}
\label{sec:kass-og-inka}

Da man i 2013 valgte at udvidde bestyrrelsen med en INKAsator, besluttede man
også at at rokkokøllerne der tidligere har tilhørt \KASS nu skulle tilhøre INKA.
Der blev lavet en ny kugleramme til \KASS som erstatning. Denne pakke har de nye
tildelinger. For at historiske dokumenter stadig virker nye versioner af denne pakke
får man forskellige piktogrammer alt efter om \texttt{gf} er sat til før eller
efter 2013. Se \vref{tab:kass-og-inka-ikon}. Det er muligt at gennemtvinge at
\KASS altid har kuglerammen ved at sætte \texttt{koelleKASS=nej}.

\begin{table}[hbt]
  \centering
  \caption{\KASS- og INKA-ikoner.}
  \label{tab:kass-og-inka-ikon}
  \newcommand{\testtext}[2]{%
    \multirow{2}{*}{#2} & \vismakro[#1]{\ikonKASS} \\
    & \vismakro[#1]{\ikonINKA} \\
  }
  \begin{tabularx}{\textwidth}{r l X Z l X}
    \testtext{}{Standard i dag} \\
    \testtext{\TKsetup{gf=2013}}{Med \texttt{gf=2013} eller før.} \\
    \testtext{\TKsetup{koelleKASS=niks,gf=2013}}{Med \texttt{gf=2013,
    koelleKASS=niks}} \\
    & \vismakro{\ikonKASSkoeller}  \\
    & \vismakro{\ikonKASSkugler}
  \end{tabularx}
\end{table}


% Disse makroer tager et valgfrit argument. Dette bliver brugt som
% parametre til \cs{includegraphics}-kaldet. Du kan altså fx skrive
% \mdframedinputminted{latex}{eksempel-pikto.tex}
% og opnå
% \begin{resultat}
%   CERMs horn er 4 cm langt: \CERMludo[width=4cm]
  
% \end{resultat}
% Hvis det valgfri argument udelades kaldes \cs{includegraphics} som
% \begin{minted}[escapeinside=||]{latex}
%   \includegraphics[height=|\meta{værdi af nøglen \optionname{ludo}}|]{CERM-ludo}
% \end{minted}

% Du vil formentlig altid ønske at skalere ned på den ene eller anden
% måde, idet grafikfilernes naturlige størrelse er ret stor. På
% side~\pageref{cha:tekniske-detaljer} er (en lysere version) af fanen
% inkluderet som vandmærke ved naturlig størrelse.

\section{Anciennitet}
\label{sec:anciennitet}

Når man når en vis \TK-alder kan det være svært at huske den rette
potens. Der er også situationer hvor noget \LaTeX-kode må forventes at
blive genbrugt (eksempelvis sangtekster), og så ville det være rart
hvis den slags blev automatisk opdateret.

\begin{minted}[escapeinside=||]{latex}
  \TKprefix{|\meta{generalforsamlingsår}|}
\end{minted}

Makroen \cs{TKprefix} tager et obligatorisk argument, som skal være et
to- eller firecifret årstal der beskriver hvornår den pågældende titel
blev opnået. Et firecifret tal (mindst~1956) bruges som det er, mens
et tocifret tal fortolkes som et år i intervallet 1956--2055. Tal
mellem 100 og 1955 udløser fejl. Som eksempel vil

\mdframedinputminted{latex}{eksempel-prefix1.tex}
give resultatet

\begin{resultat}
Dette dokument er skrevet af \TKprefix{2001}FURA, 
som også kan kalde sig \TKprefix{02}\KASS.

\end{resultat}

Præfikset beregnes relativt til \emph{årstallet} for en
generalforsamling, som man i de fleste tilfælde vil sætte til det
årstal den nuværende \textsc{best}yrelse blev valgt. Det kan enten
sættes når pakken hentes, eller vha. makroen \cs{TKsetup}. Se mere
herom i \vref{sec:konfiguration}.

I en tidligere version kunne man give \cs{TKprefix} et valgfrit
argument for lokalt at vælge et andet referenceår. Den funktionalitet
er \hyperref[sec:-og-]{fjernet af robusthedshensyn}.

Makroen
\begin{minted}{latex}
  \TKgetgf
\end{minted}
printer det årstal der pt. benyttes som reference.


\chapter{Konfiguration}
\label{sec:konfiguration}

Forskellige aspekter af \ptket kan konfigureres.

\begin{minted}[escapeinside=||]{latex}
  \TKsetup{|\meta{key-value-liste}|}
\end{minted}

Al konfiguration foregår via makroen \cs{TKsetup}. Denne tager netop
et argument, som skal være en kommasepareret liste af par af formen
  \meta{key}\texttt{=}\meta{value}.
\Vref{tab:keyval} viser de forskellige nøgler og deres
standardværdier.

% \TKsetup
\begin{table}[hbtp]
  \centering
  \caption{Nøgler og standardværdier for \cs{TKsetup}.}
  \label{tab:keyval}
  \begin{tabular}[c]{>{\ttfamily}l l}
    \multicolumn{1}{l}{Nøgle} & Standardværdi \\ \hline
    epsilon        & \cs{ensuremath}\{\cs{epsilon}\} \\
    dollar         & \cs{\$} \\
    cermC          & \cs{ensuremath}\{\cs{mathbb}\{C\}\} \\
    vcC            & \cs{ensuremath}\{\cs{mathbb}\{C\}\} \\
    ikonh          & 1.0em  \\
    eksponent      & jep \\
    % AA           & false \\
    gf             & \small (se forklaring i teksten)\\
    koelleKASS     & ja
  \end{tabular}
\end{table}

De enkelte nøgler beskrives nedenfor. Efter denne gennemgang er der et
konkret eksempel på brugen af \cs{TKsetup}.
% Nøgler der er markeret som \enquote{boolean} kan kun gives værdierne \enquote{true}
% og \enquote{false}.
Bemærk at alle ændringer er lokale til indeværende
\TeX-gruppe. Specielt vil ændringer foretaget inden i et environment
altså kun have betydning dér, hvilket kan være meget nyttigt.

\descopt{epsilon} Denne nøgle kontrollerer det tegn der bruges i
stedet for \enquote{e} i ordet RemToR når makroen \cs{RemToR} benyttes. Mange
har som en del af deres standardkode et par besværgelser der bytter om
på betydningen af \cs{epsilon} ($\epsilon$) og \cs{varepsilon}
($\varepsilon$), fordi den anden mulighed er et \enquote{bedre} epsilon. Alt
afhængig af disse indstillinger, hvilken font du benytter og hvilket
\RemToR-e du foretrækker (som altsammen er uden for \ptket{}s kontrol)
kan du måske have lyst til at sætte \optionname{epsilon}.

\descopt{dollar} Denne nøgle bestemmer hvilket tegn der bruges (to
gange) i \cs{KASS}. Standarden er, som det fremgår, et simpelt
dollartegn. Vær opmærksom på at både \cs{textdollaroldstyle}
(\textdollaroldstyle) og \cs{pounds} (\pounds) kræver
\pakkenavn{textcomp}-pakken.

\descopt{cermC} Hvis man ønsker at skrive CERM med et anderledes \enquote{C}
kan denne nøgle sættes til fx
\cs{ensuremath}\texttt{\{}\cs{mathbb}\texttt{\{C\}\}}. Du skal
selvfølgelig sørge for at hente en pakke som stiller den pågældende
font til rådighed (fx \pakkenavn{amsfonts}). Med standardværdien er
der ingen forskel på at skrive \cs{CERM} og \texttt{CERM} i ens
\LaTeX-kode.

\descopt{vcC} Tilsvarende for C'et i VC.

\descopt{C} Denne nøgle kan bruges til at sætte begge de to
ovenstående til en fælles værdi.

\descopt{ikonh} Højden som ikonerne som standard skaleres
til. Som værdi kan du bruge alt hvad der udtrykker en længde i \LaTeX,
fx \enquote{10pt}, \enquote{2.5ex}, \enquote{30mm}, eller endda udtryk som
\enquote{\cs{baselineskip}-1pt}. Hvis du bruger en relativ enhed (\enquote{em} eller
\enquote{ex}) følger piktogrammernes størrelse med størrelsen af den omgivende
tekst; dette kan fx være meget nyttigt i overskrifter. Standardværdien
på 1.0em svarer nogenlunde til højden af en tekstlinje.

\descopt{eksponent} Hvorvidt \enquote{tipeksponenter} produceret af
\cs{TKprefix} skal skrives hævet som i {\TKprefix{2001}\KASS} eller
som et tal efter T som i
{\TKsetup{eksponent=nejtak}\TKprefix{2001}{KASS}}. (Denne kontrollerer
også om \cs{TKprefix}\texttt{\{2040\}} giver \TKprefix{2040} eller
{\TKsetup{eksponent=niks}\TKprefix{2040}}.) Standardopførslen er at
eksponenter skrives hævet. Kan gives værdierne \enquote{\texttt{true}} og
\enquote{\texttt{false}}.
% \descopt{AA} Boolean. Hvis denne er sat benytter \cs{TK}-makroerne
%   dobbelt-A i stedet for bolle-Å, mens deres \texttt{*}-udgaver gør
%   det modsatte. Standardværdien er \enquote{false}.

\descopt{gf} Denne nøgle bruges til at sætte årstallet for den
generalforsamling der bruges som reference i makroen
\cs{TKprefix}. Hvis man vil have pt. gyldige titler skal denne sættes
til det årstal hvor den nuværende bestyrelse blev valgt. Hvis man ikke
i løbet af preamblen selv angiver en værdi, bruger pakken følgende
heuristik: I oktober, november og december er det indeværende år, i
alle andre måneder er det året før. Hvis dags dato er mellem den
15. og 30. september skriver pakken en advarsel til terminalen og
logfilen. Som en speciel feature kan man også angive værdien
\texttt{*}, som betyder \enquote{sæt referenceåret til det som den netop
beskrevne heuristik angiver}.

\descopt{koelleKASS} Bestemmer om \cs{ikonKASS} skal give \ikonKASSkoeller eller
\ikonKASSkugler når \texttt{gf=<2013}. Se \vref{sec:kass-og-inka}.

\section{Eksempel}
\label{sec:eksempel}

Vi vil lave et dokument som viser titler som de var i forbindelse med
50-årsjubilæet i 2006. Dette blev afholdt et par uger efter
generalforsamlingen 2006. Vi vil gerne skrive CERM som
$\mathbbm{C}$ERM, men VC skal blot skrives med normale bogstaver. Endelig
vil vi gerne benytte os af dobbeltstregede dollartegn til \KASS. Det
kan opnås ved at skrive
\mdframedinputminted{latex}{eksempel-konf.tex}
i preamblen, sammen med passende \cs{usepackage}-kald for at stille
\cs{mathbbm}-fonten og \textdollaroldstyle{} til rådighed.


\section{Fjollede bemærkninger}
\label{sec:fjoll-bemarkn}

Man kan også bruge \enquote{on}, \enquote{yes}, \enquote{ja}, \enquote{jatak}, \enquote{jep}, \enquote{off}, \enquote{no},
\enquote{nej}, \enquote{nejtak} og \enquote{niks} som værdier til en boolean nøgle (pt. er kun
\optionname{eksponent} en boolean nøgle). Man kan også helt udelade en
værdi, hvilket slår den pågældende nøgle til.

Ved misbrug af \optionname{dollar}-optionen kan man frembringe diverse ord:\bigskip

\noindent
\newcommand*{\fjolleord}[1]{#1 & \TKsetup{dollar=#1}\KASS &}
\begin{tabularx}{\textwidth}{llX}
  Hvis \optionname{dollar} er\ldots &\ldots giver \cs{KASS} \ldots &
  \ldots som betyder\\ \hline
  \fjolleord{NO} bogstavet \enquote{O} -- på den fede måde \\
  \fjolleord{OS} generisk røg \\
  \fjolleord{RENS} et vaskeri \\
  \fjolleord{RET} Diner Transportable(?) \\
  \fjolleord{SUS} kick som humanister får af at lege med grammatik \\
  \fjolleord{MEL} øøh\ldots
\end{tabularx}


\chapter{Tekniske detaljer}
\label{cha:tekniske-detaljer}

% \AddToShipoutPictureBG*{%
%   \includegraphics{FUlogo-ludo-lys}}

Der er gjort temmelig meget for at gøre denne pakke brugervenlig og
uden overraskelser; den er endda en smule intelligent. Dette kapitel
indeholder nogle tekniske bemærkninger.

\section{Navnerum}
\label{sec:navnerum}

Til interne makroer benyttes \cs{tket@} som præfiks. Dette kolliderer
ikke med nogen kendte pakker. Det samme gælder alle de offentlige
makroer.  De korte strenge \cs{TK} og \cs{VC} kunne risikere at være
brugt i anden sammenhæng, men de længere makronavne er meget
usandsynlige at finde i en ikke-\TK-relateret kontekst.

\section{* \& [ ]}
\label{sec:-og-}

Tidligere versioner af denne pakke havde \texttt{*}-varianter af
\cs{TK}-makroerne, fx \cs{TKET*}, som var den måde man kunne opnå
stavningen med AA, og desuden var der en metode til at bytte rundt på
betydningen af de to varianter (så \cs{TKET} brugte AA, mens
\cs{TKET*} brugte Å). Imidlertid er den slags makroer svære at få til
at opføre sig rigtigt i alle sammenhænge, og det blev derfor besluttet
at droppe dem til fordel for et ekstra sæt af makroer med \texttt{AA}
i enden.

En tilsvarende bemærkning angår \cs{TKprefix}: Man kunne tidligere
give et optional argument for, for et enkelt kald af makroen, at ændre
referenceåret, men \LaTeX s måde at håndtere den slags på er desværre
ikke helt robust. Derfor er også den mulighed skrottet. Så hvad man
tidligere kunne skrive som
\begin{minted}{latex}
  \TKprefix[2003]{1998}
\end{minted}
skal altså nu skrives som
\begin{minted}{latex}
  {\TKsetup{gf=2003}\TKprefix{1998}}
\end{minted}
eller, en smule kortere,
\begin{minted}{latex}
  {\TKsetgf{2003}\TKprefix{1998}}
\end{minted}
\fboxsep=0pt
(bemærk det yderste sæt $\{\}$-parenteser som holder ændringen lokal.)

\section{\TKETpdf og \ikonCERMpdf i overskrifter}
\label{sec:tket-i-overskrifter}

Det er i almindelighed ikke noget problem at bruge \cs{TK}-makroerne
og ikonerne i overskrifter som denne: Både indholdsfortegnelsen og
overskriften ser (forhåbentlig) ud som forventet. Imidlertid kan der
opstå et problem hvis man bruger pakken \pakkenavn{hyperref}. Denne
pakke laver, blandt meget andet, en elektronisk indholdsfortegnelse i
pdf-filen. I denne kan kun almindelig tekst bruges, og
hoppe-danse-bogstaverne og ikonerne tæller ikke som almindelig
tekst. Hvis man får en fejl som
\begin{minted}[fontsize=\footnotesize]{latex}
Package hyperref Warning: Token not allowed in a PDF string (PDFDocEncoding):
(hyperref)                removing `\TKET' on input line 430.
\end{minted}
og pdf-bogmærket kun viser \enquote{og i overskrifter} når man prøver
\begin{minted}{latex}
\section{\TKET og \ikonCERM i overskrifter}
\end{minted}
er løsningen er at give to
forskellige tekststrenge; en som \LaTeX{} spiser, og en som bruges i
bogmærket. Det gøres ved at skrive
\begin{minted}{latex}
\section{\texorpdfstring{\TKET}{TÅGEKAMMERET} og
  \texorpdfstring{\ikonCERM}{CERM} i overskrifter}
\end{minted}

Hvis \ptket opdager at \pakkenavn{hyperref}-pakken er i brug laves der
automagisk et antal \cs{...pdf}-makroer så ovenstående kan forkortes
til det der faktisk er brugt i denne manual:
\begin{minted}{latex}
\section{\TKETpdf og \ikonCERMpdf i overskrifter}
\end{minted}
Mere præcist betyder \enquote{opdager} at \ptket tjekker \cs{AtBeginDocument}
hvorvidt \pakkenavn{hyperref} er hentet, og handler
derefter. Rækkefølgen hvori \pakkenavn{hyperref} og \ptket hentes er
underordnet. Mere præcist betyder \enquote{et antal} at der laves
\cs{...pdf}-varianter af alle makroerne i \vref{tab:simple-makroer,tab:ikon}.
For standarddefinitionen af \cs{KASS} er det
egentlig overflødigt, men hvis \optionname{dollar} sættes til noget
besynderligt kan det være nødvendigt.

\begin{quote}
  \itshape  Disse \cs{...pdf}-makroer er kun lavet for at gøre det nemt at løse
  et problem ved blot at tilføje tre bogstaver til makrokaldet. Hvis
  du ikke har et problem skal du ikke prøve at løse det.
\end{quote}

Makroen \cs{TKurl} producerer, hvis \pakkenavn{hyperref}-pakken er i
brug, et hyperlink til \TKETs hjemmeside, \TKurl.

\subsection{To bemærkninger om \RemToRpdf}
\label{sec:bemarkn-om-remt}

Makroen \cs{RemToR} er forsynet med en lille snedighed, som er
relevant når \RemToR skrives med fed skrift. Se selv:
\mdframedinputminted{latex}{eksempel-RemToR.tex}

\begin{resultat}
\bfseries Alt dette er med fed, også \RemToR. Men den naive skrivemåde
R$\epsilon$mToR ser forkert ud.

\end{resultat}

Snedigheden består i at vi tjekker om omgivelserne er fede, og i så
fald sørger for at al eventuel matematik også skrives med fed. Det er
især relevant i overskrifter som den herover.

Den anden bemærkning angår makroen \cs{RemToRpdf}, og er altså kun
relevant for brugere af \pakkenavn{hyperref}-pakken som bekymrer sig
om pdf-bogmærker. Som det fremgår af nærværende fil kan det godt lade
sig gøre at få et rigtigt epsilon i bogmærket. Det kræver blot at
\pakkenavn{hyperref} er loadet med \optionname{unicode} som option
(man får så $\epsilon$ med \cs{textepsilon}). Makroen \cs{RemToRpdf}
er lavet til at bruge \cs{ifHy@unicode} til at afgøre om der skal
bruges \cs{textepsilon} eller blot \enquote{e}.

\section{Anciennitet og email}
\label{sec:anciennitet-og-email}

En af fordelene ved at droppe det valgfri argument til \cs{TKprefix}
var at makroen kunne gøres, i \TeX-lingo, \enquote{fully expandable}. Det
betyder at den kan bruges i ellers skrøbelige steder som fx i første
argument til \cs{href}:
\mdframedinputminted{latex}{eksempel-href.tex}
(Hold musen over teksten nedenfor for at se emailadressen som et \enquote{tooltip}.)
\begin{resultat}
\href{mailto:\TKprefix{01}FURA@TAAGEKAMMERET.dk}{Send mig en mail!}

\end{resultat}
Som en helt særlig feature ignoreres værdien af \optionname{eksponent}
i denne kontekst, så emailadressen faktisk er gyldig. Ellers ville
resultatet blive
\begin{resultat}
\edef\MitFUPrefix{\TKprefix{01}}
\href{mailto:\MitFUPrefix FURA@TAAGEKAMMERET.dk}{Send mig en mail!}

\end{resultat}

\section{Historik}
\label{sec:historik}

Den nuværende vedligeholder af pakken er \TKprefix{2002}\KASS, Rasmus
Villemoes, som nemmest kan kontaktes via emailadressen
\emaillink{KASS0203@TAAGEKAMMERET.dk}. Bugrapporter, forslag til nye
features og andre kommentarer modtages gerne.

T-shirtpiktogrammerne er oprindelig tegnet af \TKprefix{2005}PR,
Katrine Skovbo, i 2006. Efterfølgende blev de digitaliseret og
modificeret af \TKprefix{2005}FORM, Dan Beltoft, så de var brugbare i
en størrelse svarende til højden af løbende tekst. Det er disse
modificerede versioner som stilles til rådighed af denne pakke. Hvis
du har brug for den grafik der faktisk bruges til at producere
T-shirts må du kontakte \TKETs bestyrelse.

Før maj 2009 fandtes der diverse unummererede versioner af filen
\filnavn{tket.sty} rundt omkring, som definerede varierende delmængder
af makroerne \cs{TK}, \cs{TKET}, \cs{TKETs} og \cs{TKETS}. Disse
definitioner var ikke altid indbyrdes kompatible. Især genitivformerne
var et problem; makroen \cs{TKETS} kunne både give \TKETS, \TKETs og
\TKET{}s.

Det nuværende projekt startede som et forsøg på at forene,
standardisere, og udvide disse mange versioner, og, fra version 2.0,
dokumentere anvendelsen. Uddrag af historikken:

\begin{description}
\item[2.1] (2011-06-03) Stort set komplet reimplementering:
  \begin{itemize}
  \item Nu med dokumentation.
  \item Alle \texttt{*}-varianter og valgfri argumenter endegyldigt droppet.
  \item Konfiguration via \textit{key-value}-syntaks.
  \item Ikoner inkluderet.
  \item Mere robust implementation af mange interne detaljer.
  \end{itemize}
\item[1.0] (2009-06-11) Første officielle frigivelse.
\item[0.6, 0.5, 0.4, 0.3, 0.2] Diverse, i dag, irrelevante trin på vejen.
%   Den eneste
%   forskel på denne og v0.6 er versionsnummeret (og datoen).
% \item[v0.6] (2009-06-01) Den første semi-officielle frigivelse, som
%   afventer godkendelse fra nuværende BEST. Denne version ændrede
%   kravet om pakken \pakkenavn{graphicx} til \pakkenavn{graphics}, som
%   det anbefales i \pakkenavn{graphic(s|x)}-dokumentationen. Desuden
%   introduceredes præfiks-udregneren \cs{TKprefix}. Pakken tager nu et
%   valgfrit argument, som sætter det oprindelige referenceår; men hvis
%   det udelades har pakken dog en rimelig intelligent måde at gætte
%   hvilket år den nuværende BEST blev valgt (som så bliver
%   referenceåret).
% \item[v0.5] (2009-05-28) Diverse bugs i v0.4 fikset, og makroerne
%   \cs{TKxyz} blev opbygget mere intelligent. Desuden blev \cs{RemToR}
%   og \cs{KASS} gjort konfigurerbare ved at tilføje makroer
%   \cs{RemToR(var)epsilon} og \cs{KASS(var)dollar}, som brugeren kan
%   \cs{renewcommand}'e.
% \item[v0.4] (2009-05-26) Tilføjede \cmdprint{*}-varianter af alle
%   tidligere makroer. Denne version var temmelig bugfyldt pga. forkert
%   brug af \cs{xspace} sammen med \cs{@ifstar}!
% \item[v0.3] (2009-05-23) Makroerne \cs{RemToR}, \cs{KASS} og \cs{TKETs} tilføjet.
% \item[v0.2] (2009-05-21) Tilføjede krav om pakkerne
%   \pakkenavn{graphicx} og \pakkenavn{xspace}, tilføjede \cs{xspace}
%   til makrodefinitionerne.
\item[0.1] (2009-05-20) Tilføjede \cs{ProvidesPackage}. Stiller
  makroerne \cs{TK}, \cs{TKET} og \cs{TKETS} til rådighed som
  tidligere versioner.
\end{description}

\end{document}


%%% Local Variables:
%%% coding: utf-8
%%% mode: latex
%%% mode: auto-revert
%%% TeX-master: t
%%% TeX-command-extra-options: "-shell-escape"
%%% End: 


